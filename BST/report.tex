\documentclass[UTF8]{ctexart}
\usepackage{geometry, CJKutf8}
\geometry{margin=1.5cm, vmargin={0pt,1cm}}
\setlength{\topmargin}{-1cm}
\setlength{\paperheight}{29.7cm}
\setlength{\textheight}{25.3cm}
\usepackage[dvipsnames]{xcolor}
% useful packages.
\usepackage{amsfonts}
\usepackage{amsmath}
\usepackage{amssymb}
\usepackage{amsthm}
\usepackage{enumerate}
\usepackage{graphicx}
\usepackage{multicol}
\usepackage{fancyhdr}
\usepackage{layout}
\usepackage{listings}
\usepackage{float, caption}

\lstset{
    basicstyle=\ttfamily, basewidth=0.5em
}

% some common command
\newcommand{\dif}{\mathrm{d}}
\newcommand{\avg}[1]{\left\langle #1 \right\rangle}
\newcommand{\difFrac}[2]{\frac{\dif #1}{\dif #2}}
\newcommand{\pdfFrac}[2]{\frac{\partial #1}{\partial #2}}
\newcommand{\OFL}{\mathrm{OFL}}
\newcommand{\UFL}{\mathrm{UFL}}
\newcommand{\fl}{\mathrm{fl}}
\newcommand{\op}{\odot}
\newcommand{\Eabs}{E_{\mathrm{abs}}}
\newcommand{\Erel}{E_{\mathrm{rel}}}

\begin{document}

\pagestyle{fancy}
\fancyhead{}
\lhead{张子乐, 3230104237}
\chead{数据结构与算法第五次作业}
\rhead{Nov.1st, 2024}

\section{remove函数的设计思路}
\begin{lstlisting}[
    language=C++,
    basicstyle=\ttfamily,
    breaklines=true,
    commentstyle=\itshape\color{black!50!white}
]
我首先声明了一个detachMin函数,用于寻找子树的最小节点,并在原子树中删除,最后返回最小节点。
最小节点只可能存在right child,如果该节点有right child,则该节点的parent->left=该节点的right child。
该函数主要分为两种情况讨论:传入的节点t有无left child。

如果传入节点没有left child,则该节点即为最小节点。
用该节点的right child代替该节点的位置,清空该节点并返回该节点。

如果传入节点有left child,则通过while循环找到最小节点的parent节点。
while(t->left->left != nullptr) 
    t=t->left;
将该节点的right child作为该节点parent的left child,清空该节点并返回该节点。

detachMin函数代码如下:
BinaryNode *detachMin(BinaryNode *&t)
    {
        if(t==nullptr)          //传入的节点为空,则返回空
            return nullptr;
        if(t->left == nullptr) //在t没有left child的情况下,t就是最小节点
        {
            BinaryNode *minNode = t;
            t=t->right;  //若t不存在right child则为nullptr。若t存在right child,则将right child放到原先t的位置。
            minNode->right = nullptr;   //清空minNode的子节点
            return minNode;
        }
        else //t存在left child
        {
            while(t->left->left != nullptr) //找到最小节点的parent
            {
                t=t->left;
            }
            BinaryNode *minNode = t->left;
            t->left = minNode->right; //parent->left=最小节点->right,最小节点只可能存在right child,若不存在则都为nullptr
            minNode->left=nullptr;  //清空minNode的子节点
            minNode->right=nullptr;
            return minNode;
        }
    }


    借助detachMin函数我对remove函数进行了修改。
    
    首先通过不断将待删除节点的值x与t节点的值进行比较,递归寻找值为x的待删除节点。
    在找到待删除节点后,分为两种情况:待删除节点有2个child,和待删除节点有1个或0个child。
    
    如果待删除节点有2个child,通过detachMin函数找到待删除节点右子树的最小节点。
    通过minNode->left = t->left和minNode->right = t->right将它与待删除节点的2个child连上。
    最后删除待删除节点。
    
    如果待删除节点有1个或0个child:
    通过t = (t->left != nullptr) ? t->left : t->right将它与待删除节点的child连上。
    最后删除待删除节点。

    此外,我在public的remove函数中加入了判断为空的操作,若树为空仍试图删除树中的元素则会抛出异常。
    if( isEmpty( ) )
        throw UnderflowException{ };
      
    
    remove函数代码如下:
    void remove(const Comparable &x, BinaryNode * &t) 
    {
        if (t == nullptr)  //递归结束,没有找到要删除的值,什么都不做
        {
            return;  
        }
        if (x < t->element)  //比t节点元素小,在t的左子树中递归寻找x
        {
            remove(x, t->left);
        } 
        else if (x > t->element)  //比t节点元素大,在t的右子树中递归寻找x
        {
            remove(x, t->right);
        } 
        //以下操作在找到x之后实现
        else if (t->left != nullptr && t->right != nullptr) //待删除节点有两个子节点
        {  
            BinaryNode *minNode = detachMin(t->right); //找到待删除节点右子树的最小元素节点
            minNode->left = t->left;  
            minNode->right = t->right; //用右子树的最小元素节点代替待删除节点
            delete t; 
            t = minNode;
        } 
        else //待删除节点有一个或没有子节点
        {
            BinaryNode *oldNode = t;
            t = (t->left != nullptr) ? t->left : t->right; 
            delete oldNode;//用待删除节点的左节点或右节点代替待删除节点,若没有子节点则将待删除节点置为空,删除待删除节点
        }
    }
\end{lstlisting}

\section{测试的结果}
\begin{lstlisting}[
    language=C++,
    basicstyle=\ttfamily,
    breaklines=true,
    commentstyle=\itshape\color{black!50!white}
]
1.按照10,5,15,3,7,12,18,2,4的顺序在树bst中插入元素。
输出结果为:
Initial Tree:
2
3
4
5
7
10
12
15
18

2.bst.remove(33)测试待删除节点不存在的情况
输出结果为:
Tree after removing 33:
2
3
4
5
7
10
12
15
18
待删除节点不存在,不作任何操作,测试正常。

3.bst.remove(7)测试待删除节点无子节点的情况。
输出结果为:
Tree after removing 7:
2
3
4
5
10
12
15
18
元素7被删除,测试正常。

4.bst.remove(3)测试有两个子节点的情况。
输出结果为:
Tree after removing 3:
2
4
5
10
12
15
18
元素3被删除,测试正常。

5.bst.remove(2)测试有一个子节点的情况。
输出结果为:
Tree after removing 2:
4
5
10
12
15
18
元素2被删除,测试正常。

6.bst.remove(10)测试删除root节点的情况。
输出结果为:
Tree after removing 10:
4
5
12
15
18
元素10被删除,测试正常。

7.构建新树bst1,其中只有一个节点。
BinarySearchTree<int> bst1;
bst1.insert(5);
std::cout << " tree bst1:" << std::endl;
bst1.printTree();
输出结果为:
tree bst1:
5

8.bst1.remove(5)测试树只有一个root节点且将其删除的情况
输出结果为:
bst1 after removing 5:
Empty tree
该节点被删除后树为空,测试正常。

9.bst1.remove(0)测试在空树中试图删除元素的情况
输出结果为:terminate called after throwing an instance of 'UnderflowException'
在空树中删除元素,抛出异常。

综上,remove函数各项功能测试正常。
\end{lstlisting}
\end{document}

%%% Local Variables: 
%%% mode: latex
%%% TeX-master: t
%%% End: 
