\documentclass[UTF8]{ctexart}
\usepackage{geometry, CJKutf8}
\geometry{margin=1.5cm, vmargin={0pt,1cm}}
\setlength{\topmargin}{-1cm}
\setlength{\paperheight}{29.7cm}
\setlength{\textheight}{25.3cm}
\usepackage[dvipsnames]{xcolor}
% useful packages.
\usepackage{amsfonts}
\usepackage{amsmath}
\usepackage{amssymb}
\usepackage{amsthm}
\usepackage{enumerate}
\usepackage{graphicx}
\usepackage{multicol}
\usepackage{fancyhdr}
\usepackage{layout}
\usepackage{listings}
\usepackage{float, caption}

\lstset{
    basicstyle=\ttfamily, basewidth=0.5em
}

% some common command
\newcommand{\dif}{\mathrm{d}}
\newcommand{\avg}[1]{\left\langle #1 \right\rangle}
\newcommand{\difFrac}[2]{\frac{\dif #1}{\dif #2}}
\newcommand{\pdfFrac}[2]{\frac{\partial #1}{\partial #2}}
\newcommand{\OFL}{\mathrm{OFL}}
\newcommand{\UFL}{\mathrm{UFL}}
\newcommand{\fl}{\mathrm{fl}}
\newcommand{\op}{\odot}
\newcommand{\Eabs}{E_{\mathrm{abs}}}
\newcommand{\Erel}{E_{\mathrm{rel}}}

\begin{document}

\pagestyle{fancy}
\fancyhead{}
\lhead{张子乐, 3230104237}
\chead{数据结构与算法第六次作业}
\rhead{Nov.9th, 2024}

\section{remove函数的设计思路}
\begin{lstlisting}[
    language=C++,
    basicstyle=\ttfamily,
    breaklines=true,
    commentstyle=\itshape\color{black!50!white}
]
在BinarySearchTree的remove函数的基础上修改。

待删除节点有两个子节点的情况特殊,单独讨论:
在detachMin寻找右子树最小节点的过程中,在每层递归结束前调用balance函数,并返回最小节点。
从而使右子树最小节点到待删除节点之间在remove操作后维持平衡。
右子树最小节点的右子树remove前后平衡性不变。

detachMin函数如下:
BinaryNode *detachMin(BinaryNode *&t)
{
    if(t==nullptr)
        return nullptr;
    if(t->left == nullptr)//t没有left child的情况下t就是最小节点
    {
        BinaryNode *minNode = t;
        t=t->right; 
        balance(t);//平衡t节点
        minNode->right = nullptr;
        return minNode;
    }
    else //t存在left child
    {   
        BinaryNode *minNode=t->left;
        if(t->left->left == nullptr)
        {
            t->left = minNode->right;
            minNode->left=nullptr;
            minNode->right=nullptr;
        }
        else
        {
          minNode=detachMin(t->left);
        }
        balance(t->left);//平衡当前节点的左子树
        return minNode;
    }
}

在所有情况下,采用递归的方式寻找待删除节点。
在每层递归结束前调用balance函数,使待删除节点到根节点之间在remove操作后维持平衡。

\end{lstlisting}
\end{document}

%%% Local Variables: 
%%% mode: latex
%%% TeX-master: t
%%% End: 
