\documentclass[UTF8]{ctexart}
\usepackage{geometry, CJKutf8}
\geometry{margin=1.5cm, vmargin={0pt,1cm}}
\setlength{\topmargin}{-1cm}
\setlength{\paperheight}{29.7cm}
\setlength{\textheight}{25.3cm}

% useful packages.
\usepackage{amsfonts}
\usepackage{amsmath}
\usepackage{amssymb}
\usepackage{amsthm}
\usepackage{enumerate}
\usepackage{graphicx}
\usepackage{multicol}
\usepackage{fancyhdr}
\usepackage{layout}
\usepackage{listings}
\usepackage{float, caption}

\lstset{
    basicstyle=\ttfamily, basewidth=0.5em
}

% some common command
\newcommand{\dif}{\mathrm{d}}
\newcommand{\avg}[1]{\left\langle #1 \right\rangle}
\newcommand{\difFrac}[2]{\frac{\dif #1}{\dif #2}}
\newcommand{\pdfFrac}[2]{\frac{\partial #1}{\partial #2}}
\newcommand{\OFL}{\mathrm{OFL}}
\newcommand{\UFL}{\mathrm{UFL}}
\newcommand{\fl}{\mathrm{fl}}
\newcommand{\op}{\odot}
\newcommand{\Eabs}{E_{\mathrm{abs}}}
\newcommand{\Erel}{E_{\mathrm{rel}}}

\begin{document}

\pagestyle{fancy}
\fancyhead{}
\lhead{张子乐, 3230104237}
\chead{数据结构与算法第四次作业}
\rhead{Oct.19th, 2024}

\section{测试程序的设计思路}


我首先声明了一个printlist()函数,用于打印list中的所有元素,如果list为空则输出empty list

代码如下:

template<typename T>


void printlist(const List<T>\& list)//printlist函数方便后续测试


\{   

    if(list.empty())
    
        std::cout << "empty list" << std::endl;
        
    else
    
    \{
    
    for(const auto\& i:list)
   
    \{

        std::cout << i << " ";

    \}

    std::cout << std::endl;

    \}

\}

1.为了测试默认构造函数 List(),我创建了list1并打印

代码如下:

List<int> list1;//测试默认构造函数 List()

    std::cout << "this is list1:";

    printlist(list1);

测试结果为:this is list1:empty list

2.为了测试初始化列表构造函数List(std::initializer\_list<Object> il),我用list2=\{1,3,5,7,9\}的方式创建了list2并打印

代码如下:

 List<int> list2=\{1,3,5,7,9\};//测试初始化列表构造函数List(std::initializer\_list<Object> il)

    std::cout << "this is list2:";

    printlist(list2);

测试结果为:this is list2:1 3 5 7 9

3.为了测试拷贝构造函数List(const List \&rhs),我用list3(list2)的方式创建list3并打印

代码如下:

 List<int> list3(list2);//测试拷贝构造函数List(const List \&rhs)
   
    std::cout << "this is list3:";

    printlist(list3);

测试结果为:this is list3:1 3 5 7 9

4.为了测试赋值运算符List \&operator=(List copy),我创建list4并使list4=list2,随后打印list4

代码如下:

List<int> list4;//测试赋值运算符List \&operator=(List copy)

    list4=list2;

    std::cout << "this is list4:";

    printlist(list4);

测试结果为:this is list4:1 3 5 7 9

5.为了测试移动构造函数List(List \&\&rhs),我用list5(std::move(list3))的方式创建list5并打印list5和list3

代码如下:

List<int> list5(std::move(list3));//测试移动构造函数List(List \&\&rhs)

    std::cout << "this is list5:";

    printlist(list5);

    printlist(list3);//list3应为空

测试结果为:this is list5:1 3 5 7 9

empty list

6.为了测试size()函数,我输出了list5.size()的结果,list5有5个元素

代码如下:

 std::cout << "size of list5:" << list5.size() << std::endl;//测试size()函数

测试结果为:size of list5:5

7.我用空链表list3和有5个元素的链表list5测试empty()函数,如果list3为空,则输出list3 is empty,如果list5不为空,则输出list5 is not empty

代码如下:


if(list3.empty())//测试empty()函数

        std::cout << "list3 is empty" << std::endl;

    if(!(list5.empty()))

        std::cout << "list5 is not empty" << std::endl;

测试结果为:list3 is empty

list5 is not empty

8.为了测试front()和back()函数,我用临时变量front存储list5.front(),用临时变量back存储list5.back(),list5的第一个数据为1,最后一个数据为9

代码如下:


int front=list5.front();
    
int back=list5.back();
    
std::cout << "this is front:" << front << std::endl;
    
std::cout << "this is back:" << back << std::endl;

测试结果为:this is front:1

this is back:9

9.为了测试在头部插入函数push\_front(左值版本和右值版本),用list5.push\_front(std::move(a))测试右值引用版本,用list5.push\_front(10)测试左值引用版本,相当于在链表头部插入2个10
    
代码如下:

    int a=10;//测试在头部插入函数push\_front(左值版本和右值版本)
    
    list5.push\_front(std::move(a));
    
    list5.push\_front(10);
    
    std::cout << "this is list5 after push\_front:";
    
    printlist(list5);

    测试结果为:
   
    this is list5 after push\_front:10 10 1 3 5 7 9

10.为了测试在尾部插入函数push\_back(左值版本和右值版本),用list5.push\_back(std::move(a))测试右值引用版本,用list5.push\_back(10)测试左值引用版本,相当于在链表尾部插入2个10
    
    代码如下:

    list5.push\_back(std::move(a));//测试在尾部插入函数push\_back(左值版本和右值版本)
    
    list5.push\_back(10);
    
    std::cout << "this is list5 after push\_back:";
    
    
    printlist(list5);

    测试结果为:this is list5 after push\_back:10 10 1 3 5 7 9 10 10

11.测试pop\_front(),删除list5头部第一个数据10

代码如下:

    list5.pop\_front();//测试pop\_front()
    
    std::cout << "this is list5 after pop\_front:";
    
    printlist(list5);

测试结果:this is list5 after pop\_front:10 1 3 5 7 9 10 10

12.测试pop\_back(),删除list5尾部第一个数据10

代码如下:

list5.pop\_back();//测试pop\_back()

std::cout << "this is list5 after pop\_back:";

printlist(list5);

测试结果:this is list5 after pop\_back:10 1 3 5 7 9 10

13.为了测试insert()函数的左值引用版本,在list5头部第二个位置插入8

代码如下:

    list5.insert(++list5.begin(),8);//测试insert()函数(左值版本)
    
    std::cout << "this is list5 after insert8:";
    
    printlist(list5);

测试结果:this is list5 after insert8:10 8 1 3 5 7 9 10

14.为了测试insert()函数的右值引用版本,在list5头部第二个位置再插入1个6

代码如下:

    int b=6;
    
    list5.insert(++list5.begin(),std::move(b));
    
    std::cout << "this is list5 after insert6:";
    
    printlist(list5);   

测试结果:this is list5 after insert6:10 6 8 1 3 5 7 9 10

15.测试erase(iterator itr)删除指定节点,删除list5头部第二个数据6

代码如下:

    
    list5.erase(++list5.begin());//测试erase(iterator itr)删除指定节点
    
    std::cout << "this is list5 after erase:";
    
    printlist(list5);

测试结果:this is list5 after erase:10 8 1 3 5 7 9 10

16.测试erase(iterator from, iterator to)删除范围节点,将list5从头部第二个节点到倒数第二个节点的数据全部删除
    
代码如下:
    
    list5.erase(++list5.begin(),--list5.end());//测试erase(iterator from, iterator to)删除范围节点
    
    std::cout << "this is list5 after erase:";
    
    printlist(list5);

测试结果:this is list5 after erase:10 10

17.为了测试是否深度复制,更改list2的数据将头部第一个数据删除,观察list4是否因此改变.如list4不变,则说明是深度复制

    代码如下:
    
    list2.pop\_front();
    
    std::cout << "this is list2 after pop\_front:" ;
    
    printlist(list2);
    
    std::cout << "this is list4 after pop\_front list2" ;
    
    printlist(list4);

测试结果:this is list2 after pop\_front:3 5 7 9

this is list4 after pop\_front list2:1 3 5 7 9

18.为了测试连续复制,新建一个空链表list6,用list6=list3=list2将list2的值(3,5,7,9)传给list3和list6

代码如下:

    List<int> list6;
    
    list6=list3=list2;
    
    std::cout << "this is list3:" ;
    
    printlist(list3);
    
    std::cout << "this is list6:" ;
    
    printlist(list6);

测试结果:this is list3:3 5 7 9

this is list6:3 5 7 9

19.为了测试clear()函数,将list4清空

代码如下:

list4.clear();//测试clear()函数

std::cout << "this is list4:";

printlist(list4);

测试结果:this is list4:empty list

\section{测试的结果}

编译运行List.cpp后终端输出为:

this is list1:empty list

this is list2:1 3 5 7 9

this is list3:1 3 5 7 9

this is list4:1 3 5 7 9

this is list5:1 3 5 7 9

empty list

size of list5:5

list3 is empty

list5 is not empty

this is front:1

this is back:9

this is list5 after push\_front:10 10 1 3 5 7 9

this is list5 after push\_back:10 10 1 3 5 7 9 10 10

this is list5 after pop\_front:10 1 3 5 7 9 10 10

this is list5 after pop\_back:10 1 3 5 7 9 10

this is list5 after insert8:10 8 1 3 5 7 9 10

this is list5 after insert6:10 6 8 1 3 5 7 9 10

this is list5 after erase:10 8 1 3 5 7 9 10

this is list5 after erase:10 10

this is list2 after pop\_front:3 5 7 9

this is list4 after pop\_front list21 3 5 7 9

this is list3:3 5 7 9

this is list6:3 5 7 9

this is list4:empty list

测试结果一切正常。

我用 valgrind 进行测试,发现没有发生内存泄露。

\section{(可选)bug报告}

我发现了两个bug,触发条件如下:


\begin{enumerate}
  \item bug1:
  
  \item 首先,list3.empty()能通过编译且返回值为true
    
  \item 然后我试图对空链表list3进行insert,push\_front,push\_back,pop\_front,pop\_back等操作,
    
  \item 此时发现编译报错Segmentationfault
    
  \item 据我分析,它出现的原因是:list3在list5(std::move(list3))中被析构,size为0,head和
    
    tail都为nullptr,即list3已经不存在了,因此无法对其进行插入和删除等操作;而empty()
    
    函数的判断标准为size是否为0,因此认为list3存在且为空链表;一个正常初始化的空链表的
    
    head指向tail,tail指向head,list3不符合这个标准,因此调用empty()函数时应返回异常,
    
    而不是返回true。
\end{enumerate}   

\begin{enumerate}

  \item bug2:
  
  \item 首先,lis1是一个空链表
   
  \item 然后我对list1做pop\_front()和pop\_back()操作
  
  \item 此时发现编译报错Segmentation fault
   
  \item 据我分析,它出现的原因是:list1是一个空链表,对其做删除头部和尾部第一个数据的操作是没有意义的;当对空链表做这些操作时应返回异常。
   
\end{enumerate}



\end{document}

%%% Local Variables: 
%%% mode: latex
%%% TeX-master: t
%%% End: 
